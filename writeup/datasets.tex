\section{Datasets}
\label{sec:datasets}
\subsection{Introduction}
\label{subsec:datasets_intro}

To accurately predict solar wind parameters, it is imperative to use
datasets that contain precise measurements and a cadence high enough to
capture both short and long-term trends. In this study, we make use of
the OMNI COHO dataset with hourly cadence \citep{cdaweb_omnicoho1hr}.
This dataset is formulated in the radial-tangential-normal (RTN)
coordinate system, meaning spatial correlations which would otherwise
would be present are eliminated, leaving only temporal correlations
\citep{owens_riley_2017}. In this study, we focus on radial magnetic field
strength (B$_r$) and radial wind velocity (v$_r$).

\subsection{Solar cycles}
\label{subsec:solar_cycles}

In order to ensure our training, validation and testing datasets contain
similar information for predictions to be accurate, we slice the data by
solar cycle. Cycles 21 - 22 are used in the training dataset, cycle 23 is
used for validation, and cycle 24 is used for testing. Table
\ref{table:cycle_checks} shows the number of points in each cycle, the
date ranges of these cycles, and how many NaN points are in each.

\begin{table}[h]
\begin{tabular}{lllrrr}
\toprule
Cycle &       Start &         End &  Total Points &  \% NaN (B$_r$) &  \% NaN (v$_r$) \\
\midrule
21  &  1976-03-01 &  1986-09-01 &      92073 &       35.91 &      33.94 \\
22  &  1986-09-01 &  1996-08-01 &      86937 &       49.19 &      50.21 \\
23  &  1996-08-01 &  2008-12-01 &     108131 &        0.20 &       0.35 \\
24  &  2008-12-01 &  2019-12-01 &      96418 &        0.11 &       0.15 \\
\bottomrule
\end{tabular}
\label{table:cycle_checks}
\caption{Percentage of points that are NaN in each solar cycle.}
\end{table}

% Would be good to make a histogram here to see how points are distributed
% in time: Are they random or are they in specific times in solar cycles?

